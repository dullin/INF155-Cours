% Autogenerated translation of ex6.md by Texpad
% To stop this file being overwritten during the typeset process, please move or remove this header

\documentclass[12pt]{book}
\usepackage{graphicx}
\usepackage{fontspec}
\usepackage[utf8]{inputenc}
\usepackage[a4paper,left=.5in,right=.5in,top=.3in,bottom=0.3in]{geometry}
\setlength\parindent{0pt}
\setlength{\parskip}{\baselineskip}
\setmainfont{Helvetica Neue}
\usepackage{hyperref}
\pagestyle{plain}
\begin{document}

\hrule
title: INF155 - Les tableaux
date: Exercice 6
pandocomatic\_:

\section*{    use-template: handout}

\chapter*{Question 1:}

Écrire un programme où vous: 

Déclarez un tableau de 40 entiers; 
Initialisez toutes les valeurs de votre tableau à 0;
Stockez, dans chaque case de votre tableau, une valeur aléatoire entre 0 et 100.

\chapter*{Question 2:}

Nous souhaitons maintenant créer un sous-programme remplir\_aleatoire qui, remplit les cases d'un tableau reçu en paramètre de valeurs aléatoires entre 0 et 100. Votre fonction doit avoir le prototype suivant:  

\textasciitilde{}\textasciitilde{}\textasciitilde{}c
void remplir\emph{aleatoire(int tableau[], int nb}elements); 
\textasciitilde{}\textasciitilde{}\textasciitilde{}

Ensuite, écrivez une fonction test\emph{remplir}aleatoire qui teste votre fonction remplir\emph{aleatoire. La fonction de test doit vérifier le fonctionnement de la fonction testée. Notamment, elle doit tester le fonctionnement normal (il y'a bien des valeurs aléatoires qui sont stockées dans le tableau) ainsi que le fonctionnement aux extrêmes (qu'arrive-t-il si nb}elements=0? si nb\emph{elements$<$0? si nb}elements=la taille maximale du tableau?). 

La fonction test doit avoir le prototype ci-dessous et renvoyer une valeur vraie si le test s'est bien déroulé, ou fausse sinon. 
int test\emph{remplir}aleatoire(void); 
Faites appel à votre fonction test depuis votre main. 

\chapter*{Question 3:}

Écrire une fonctions qui retourne le plus petit élément d'un tableau d'entiers, ainsi que l'indice de ce plus petit élément. Vous devez également écrire la fonction qui effectuera le test de votre sous-programme. 
Nous procéderons en deux étapes: \\ Dans la première étape, nous nous préoccupons uniquement de retourner la plus petite valeur du tableau. Si le tableau est vide, vous devez retourner la valeur de INT\_MAX (voir note ci-dessous). Ci-dessous, le prorotype de votre fonction que vous devez écrire, ainsi que la fonction de test correspondante.:  

\textasciitilde{}\textasciitilde{}\textasciitilde{}c
int trouver\emph{plus}petit\emph{element(int tab[], int nb}elements);
\textasciitilde{}\textasciitilde{}\textasciitilde{}

Ensuite, modifiez votre fonction et votre fonction de test pour retourner également l'indice du plus petit élément. Compte tenu qu'il n'est pas possible de retourner plus d'une valeur, vous utiliserez un pointeur. Si la plus petite valeur existe plus d'une fois dans la fonction, vous devez retourner l'indice de sa dernière occurence. 
Ci-dessous, le nouveau prototype de la fonction: 

\textasciitilde{}\textasciitilde{}\textasciitilde{}c
int trouver\emph{plus}petit\emph{element(int tab[], int nb}elements, int *plus\emph{petit}indice);
\textasciitilde{}\textasciitilde{}\textasciitilde{}

Note: La librairie limits.h vous fournit des constantes de précompilation qui vous donnent les valeurs minimales et maximales de chaque type de données de base. Ainsi, la constante INT\emph{MAX vous donne la valeur maxilamale que peut prendre un entier signé. La constante INT}MIN vous donne la plus petite valeurs que peut prendre un entier signé.

\chapter*{Question 4:}

Écrire une fonction (et la fonction de test correspondante) qui compte le nombre d'occurrences d'un élément dans un tableau. \\Votre fonction doit avoir le prototype suivant: 
int compter\emph{occurences(int tableau[], int nb}elts, int elt\emph{a}trouver); 

\chapter*{Question 5:}

Écrire une fonction (et la fonction de test correspondante) qui recherche un élément dans un tableau. Si l'élément a été trouvé dans le tableau, votre fonction doit retourner l'indice de l'élément. Sinon, votre fonction doit retourner la valeur -1. Si l'élément se trouve plus d'une fois dans le tableau, votre fonction doit renvoyer l'indice de sa première occurrence dans le tableau. 
Pour cette question, vous devez, vous-même, écrire le prototype de la fonction.

\chapter*{Question 6:}

Écrire une fonction (et sa fonction de test) qui copie un tableau dans un autre. Votre fonction reçoit en argument le tableau source et le tableau de destination, ainsi que les paramètres suivants: 

nb\emph{elts}source: Le nombre d'éléments effectifs du tableau source;
max\emph{elts}dest: La taille du tableau destination (i.e. nombre maximal d'éléments qu'il peut contenir). 
Votre fonction doit d'abord s'assurer que la copie est possible et elle doit renvoyer une valeur vraie si la copie a bien été effectuée, ou une valeur fausse sinon. Si le tableau destination est plus grand que le nombre de cases à copier, vous devez mettre des valeurs nulles dans les cases excédentaires.

Prototype de la fonction:

\textasciitilde{}\textasciitilde{}\textasciitilde{}c
int copier\emph{tableau(int tab}source[], int nb\emph{elts}source, int tab\emph{dest[], int max}elts\_dest); 
\textasciitilde{}\textasciitilde{}\textasciitilde{}

\chapter*{Question 7:}

Écrire une fonction, ainsi que la fonction de test correspondante, qui renverse le contenu d'un tableau : le premier élément est échangé avec le dernier, le 2e élément est échangé avec l'avant dernier, etc.

Votre fonction aura le prototype suivant:

\textasciitilde{}\textasciitilde{}\textasciitilde{}c
int renverser\emph{tableau(int tableau[], int nb}elets); 
\textasciitilde{}\textasciitilde{}\textasciitilde{}

\end{document}
